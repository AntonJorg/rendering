\section{Method}

This project implements two procedural materials: a wood material and a marble material. Both are defined analytically in 3D space and evaluated at shading time, avoiding texture lookups. The materials are based on geometric projections, trigonometric modulation, and multi-octave noise.

\subsection{Procedural Wood}

The wood material models a cylindrical tree trunk with growth rings and longitudinal grain.

\paragraph{Log* Coordinate System (*the tree part, not the function)}
A central axis is defined by a point $\mathbf{p}_0$ and a unit direction $\mathbf{d}$. For a surface position $\mathbf{p}$, the coordinate along the trunk is
\[
t = (\mathbf{p} - \mathbf{p}_0) \cdot \mathbf{d}.
\]
The closest point on the axis is $\mathbf{q} = \mathbf{p}_0 + t \mathbf{d}$, and the radial vector in the plane orthogonal to the trunk is
\[
\mathbf{r} = \mathbf{p} - \mathbf{q}, \qquad r = \|\mathbf{r}\|.
\]

An orthonormal basis $(\mathbf{u}, \mathbf{v})$ spanning the plane orthogonal to $\mathbf{d}$ is constructed, allowing an angular coordinate around the trunk:
\[
\theta = \operatorname{atan2}(\mathbf{r}\cdot\mathbf{v},\; \mathbf{r}\cdot\mathbf{u}).
\]

\paragraph{Growth Rings}
Growth rings are modeled as a periodic function of the radial distance. To reproduce wider rings near the center and tighter rings near the bark, a nonlinear phase function is used:
\[
\phi(r) = f \left(r + \alpha r^2\right),
\]
where $f$ is a base ring frequency and $\alpha$ controls the tapering of ring width. The raw ring signal is then
\[
R(r) = \tfrac{1}{2}\left[1 + \sin(\phi(r))\right].
\]
A smooth thresholding operation is applied to sharpen the contrast between earlywood and latewood regions.

\paragraph{Domain Warping}
To avoid perfectly concentric rings, the radial coordinate is perturbed using low-frequency noise:
\[
\tilde{r} = r + \beta_1 N_1(\mathbf{p}) + \beta_2 N_2(\mathbf{p}),
\]
where $N_1$ and $N_2$ are noise functions at different scales and $\beta_i$ control the warp amplitude. This technique, known as domain warping, produces natural irregularities in the ring structure.

\paragraph{Color Composition}
Two base colors are defined for earlywood and latewood. The final diffuse color is computed by mixing these colors according to the ring signal and modulating the result by the grain term and an additional low-frequency tint variation.

\subsection{Procedural Marble}

The marble material is based on sinusoidal bands distorted by turbulent noise.

\paragraph{Vein Direction}
A unit vector $\mathbf{d}$ defines the dominant vein direction. For a point $\mathbf{p}$, a coordinate along this direction is
\[
x = (\mathbf{p} - \mathbf{p}_0)\cdot\mathbf{d}.
\]

\paragraph{Turbulence and Bands}
A multi-octave noise function $T(\mathbf{p})$ is evaluated and remapped to $[-1,1]$ to produce turbulence. The marble pattern is generated by modulating a sinusoid with this turbulence:
\[
M(\mathbf{p}) = \tfrac{1}{2}\left[1 + \sin\left(\omega x + \gamma T(\mathbf{p})\right)\right],
\]
where $\omega$ controls band spacing and $\gamma$ controls vein distortion. A smooth threshold is applied to emphasize veins.

\paragraph{Color Mapping}
Two colors are defined for the base stone and the veins. The final diffuse color is obtained by interpolating between them using the processed band signal.

\subsection{Noise Functions}

Both materials rely on procedural noise for natural variation.

\paragraph{Value Noise}
A scalar value noise function is defined on a 3D grid. For a point $\mathbf{p}$, the surrounding lattice points are assigned pseudo-random values via a hash function. Trilinear interpolation with a smoothstep kernel produces a continuous noise field:
\[
N(\mathbf{p}) = \text{lerp}_{xyz}\left(h(\lfloor\mathbf{p}\rfloor + \mathbf{i})\right),
\]
where $h$ is a hash function and $\mathbf{i}\in\{0,1\}^3$.

\paragraph{Fractal Brownian Motion}
To obtain richer structure, multiple octaves of noise are summed:
\[
\text{fBm}(\mathbf{p}) = \sum_{k=0}^{K-1} a_k\, N(2^k \mathbf{p}),
\]
with amplitudes $a_k$ decreasing geometrically. This produces scale-invariant, natural-looking variation used for both domain warping and fine detail.
